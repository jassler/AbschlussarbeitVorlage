% Überschriften, Inhaltsverzeichnisse, etc. auf Deutsch
\usepackage[ngerman]{babel}

% Text unter Abbildungen, Tabellen, etc.
\usepackage[indention=1cm,format=plain,textformat=simple,labelformat=simple,labelfont=footnotesize,textfont=footnotesize,position=bottom]{caption}
\captionsetup[table]{position=below}

% Schönere Tabellen
\usepackage{booktabs}

% Bessere Mathe-Funktionalitäten
% Fixt ein paar Mathe-Renderprobleme bei amsmath
\usepackage{mathtools}

% Hier um Spalten einer Tabelle fett zu machen
\usepackage{array}
\usepackage{tabu}

% Titel, Author, Datum kann mit \thetitle, \theauthor, \thedate aufgerufen werden
\usepackage{titling}

% Kopf- und Fußzeile
\usepackage{fancyhdr}
% Kopf- und Fußzeile für Abstrakt, Deklaration und Inhaltsverzeichnis
\pagestyle{fancy}
\fancyhf{} % alles bereinigen
\fancyhead[L]{\nouppercase{\leftmark}} % Kopfzeile links
\fancyhead[C]{} %zentrierte Kopfzeile
\fancyhead[R]{\thepage} %Kopfzeile rechts
\setlength{\headheight}{15pt}
\renewcommand{\headrulewidth}{0.4pt} %obere Trennlinie

\fancypagestyle{plain}{%
  \fancyhf{} %alle Kopf- und Fußzeilenfelder bereinigen
  \fancyhead[R]{\thepage} %Kopfzeile rechts
}


% Layout bzw. Text für Kapitelüberschriften
\usepackage{titlesec}
\titleformat{\chapter}[hang]{\normalfont\huge\bfseries}{\thechapter\ }{1em}{}

% Listing Captions
\usepackage{floatrow}
\DeclareNewFloatType{listing}{placement=H, fileext=lsts, name=}
\captionsetup{options=listing}
\renewcommand{\thelisting}{\thechapter.\arabic{listing}}
\makeatletter
\@addtoreset{listing}{chapter}
\makeatother
\newcommand{\codeblock}[2]{\refstepcounter{listing}\label{#1}Listing~\thelisting: #2\vspace{-10pt}}
\newcommand{\lstbeg}[2]{\begingroup\begin{center}\codeblock{#1}{#2}\phantomsection\addcontentsline{lsts}{figure}{Listing \thelisting: #2}\end{center}}
\newcommand{\lstend}{\endgroup}

% Inhaltsverzeichnis für Listings
\makeatletter
%\newcommand\listingname{Listings}
\newcommand\listoflistings{%
  \chapter*{Listings}\@starttoc{lsts}}
\makeatother

% 1,5-facher Zeilenabstand
\usepackage[onehalfspacing]{setspace}
% Zeileneinzug nach jedem Paragraphen
\setlength{\parindent}{15pt}
% Zusätzlicher Zeilenabstand nach jedem Paragraphen
\setlength{\parskip}{0pt}

% Zitierformat
\usepackage{natbib}
\setcitestyle{numbers}
\setcitestyle{square}

% maketitle soll nicht angezeigt werden, Titelseite wird mit frontpage.tex generiert
\AtBeginDocument{\let\maketitle\relax}
